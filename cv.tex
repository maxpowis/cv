% !TEX TS-program = lualatex

\documentclass[a4paper,9pt]{article}

\usepackage[usenames,dvipsnames]{xcolor}
\usepackage{libertine}
\usepackage{fontawesome}
\usepackage{longtable}
\usepackage[cm]{fullpage}

% headers and footers
\usepackage{fancyhdr}
\usepackage{lastpage}
\pagestyle{fancy}
\fancyfoot[L]{Max Powis}
\fancyfoot[C]{Page \thepage\ of \pageref*{LastPage}}
\fancyfoot[R]{\today}
\renewcommand{\footrulewidth}{0.4pt}% default is 0pt
\fancyhead{}
\renewcommand{\headrulewidth}{0pt}

% StackOverflow-like tags
% https://tex.stackexchange.com/a/311949/142692
% https://tex.stackexchange.com/questions/387499/how-to-create-a-border-that-looks-like-a-tag
\usepackage{tikz}
\definecolor{tagbg}{RGB}{225,236,244}
\definecolor{tagtxt}{RGB}{88,115,159}
\newcommand{\sotag}[1]{\tikz[baseline]{\node[anchor=base, rounded corners=0.5ex, text height=1.5ex, text depth=.25ex, fill=tagbg, draw=tagbg, text=tagtxt] {#1};}}
\newcommand{\sotagtech}[1]{\tikz[baseline]{\footnotesize\node[anchor=base, rounded corners=0.5ex, text height=1.5ex, text depth=.25ex, fill=tagbg, draw=tagbg, text=tagtxt] {#1};}}

% Helpers for adding job entries
\usepackage[english]{babel}
\usepackage[en-GB,calc,useregional,useregional=numeric,datesep=/]{datetime2}
\DTMnewdatestyle{shortmonth}{%
 \renewcommand{\DTMdisplaydate}[4]{%
  \DTMshortmonthname{##2} \number##1 }%
 \renewcommand{\DTMDisplaydate}{\DTMdisplaydate}%
}
\newcommand{\displayshortmonth}[1]{%
{% group actions
  \DTMsetdatestyle{shortmonth}%
  \DTMsavedate{mydate}{#1}\DTMUsedate{mydate}%
}%
}%
\newcount\datediffdays
\newcounter{diffdays}
\newcommand{\setdatediffdays}[2]{%
  \DTMsavedate{startdate}{#1}%
  \DTMsavedate{enddate}{#2}%
  \DTMsaveddatediff{enddate}{startdate}{\datediffdays}%
  \setcounter{diffdays}{\number\datediffdays}%
  \ifnum\value{diffdays}<0
    \setcounter{diffdays}{-\value{diffdays}}%
  \fi
}
\newcommand{\displaydaysdiff}[2]{%
  \setdatediffdays{#1}{#2}%
  \thediffdays\space day\ifnum\value{diffdays}>1s
}
\usepackage{calc}
\newcounter{diffyears}
\newcounter{diffmonths}
\newcommand{\displaymonthsdiff}[2]{%
  \setdatediffdays{#1}{#2}%
  \setcounter{diffyears}{\value{diffdays}/365}%
  \setcounter{diffdays}{\value{diffdays}-365*\value{diffyears}}%
  \setcounter{diffmonths}{\value{diffdays}/30}%
  \setcounter{diffdays}{\value{diffdays}-30*\value{diffmonths}}%
  %
  \ifnum\value{diffyears}=0
  \else
    \thediffyears\space year\ifnum\value{diffyears}>1s\fi
    \ifnum\value{diffmonths}>0, \fi
  \fi
  \ifnum\value{diffmonths}=0
  \else
    \thediffmonths\space month\ifnum\value{diffmonths}>1s\fi
  \fi
}
\newcommand{\joblog}[5]{\textsc{\displayshortmonth{#4} -- \displayshortmonth{#5}} & \large\sffamily \textbf{#1} at \textbf{#2}, {#3}\\\textit{(\displaymonthsdiff{#4}{#5} )}}
\DTMsavenow{now}
\newcommand{\joblogcurrent}[4]{\joblog{#1}{#2}{#3}{#4}{\DTMfetchyear{now}-\DTMfetchmonth{now}-\DTMfetchday{now}}}
\newcommand{\sep}{\multicolumn{2}{c}{}\\}

% tweak the url colors
\usepackage{hyperref}
\definecolor{linkcolor}{rgb}{0,0.2,0.6}
\hypersetup{colorlinks,breaklinks,urlcolor=linkcolor, linkcolor=linkcolor}

% nicer-looking section titles
\usepackage{titlesec}
\titleformat{\section}{\Large\scshape\raggedright}{}{0em}{}[\titlerule]
\titlespacing{\section}{0pt}{1em}{3pt}

\begin{document}

% --------------------TITLE-------------
\par{\centering
		{\Huge \textsc{Max Powis}
	}\bigskip\par}

\hrule
\vspace{0.5em}
\begin{tabular}{rl}
  \textsc{Phone:}     & +32 499 673767\\
  \textsc{Email:}     & \href{mailto:max@pow.is}{max@pow.is}\\
  \textsc{Socials:}   & \faFirefox{} \href{https://max.pow.is}{https://max.pow.is} 
                      | \faTwitter{} \href{https://twitter.com/maxpowis}{twitter} 
                      | \faLinkedin{} \href{https://www.linkedin.com/in/maxpowis/}{linkedin}
%                      | \faGithub{} \href{https://github.com/maxpowis}{github}
%                      | \faStackOverflow{} \href{https://stackoverflow.com/users/8205/igal-tabachnik}{stackoverflow}
\end{tabular}

\section{Summary}
\begin{tabular}{p{0.9\textwidth}}
  I'm a data engineer with deep functional and technical expertise in a wide range of data matters: data architecture, data modeling, data integration, system integration, reference data, master data management and business analytics. \\\\

  Interests: \sotag{data-modeling} \sotag{data-architecture} \sotag{data-integration} \sotag{data-warehouse-automation} \end{tabular}

\section{Companies I have worked for}
\begin{longtable}{r|p{0.72\textwidth}}
  \joblogcurrent{Freelance Data Expert}{maxPowX}{Brussels}{2019-01-07}
    &Consultant in data architecture, data modeling, data governance, data integration, system integration, reference data, master data management and business analytics.\\\sep
  
  %\hline
  %\multicolumn{2}{r}{\footnotesize\itshape (abbreviated work history below, see LinkedIn profile for full details)}\\\sep

  \joblog{Senior Data Engineer}{dFakto}{Brussels}{2012-10-01}{2019-01-04}
    &Having deep interest in data modeling and data management, I have been devoted to smoothening business processes and providing governance bodies with reliable information through advanced data integration techniques.\\&\\
    &Drove the organization towards more agile delivery through methodology mashups fit for consulting firms.\\\sep
  
  \joblog{Analyst Programmer}{Accenture Technology Solutions}{Brussels}{2006-11-27}{2012-09-30}
    &Integrated enterprise applications (EAI) for major companies active in the telecommunication and banking industries.\\&\\
    &Extensive experience with offshore application delivery centers in Bratislava and Bangalore.\\&\\
    &Took part in the application design, implementation, test and delivery management of the Java technology based products.\\\sep
  
\end{longtable}

\section{Missions I have carried out}
\begin{longtable}{r|p{0.72\textwidth}}
  \joblogcurrent{Data Governance Manager}{Belgian Telco}{Brussels}{2019-01-11}
    &\sotagtech{Collibra} \sotagtech{REST API} \sotagtech{Swagger} \sotagtech{SOAP UI}\\&\\
    &As a Data Model governenca capablity owner, I was responsible to define the capability functions and implementations guidelines towards a new way of working.\\&\\
    &I was also responsible for the implementation of the glossary capability for the BI Community from design to delivery, including awareness and training session for 130+ people.\\&\\
    &Strong from past experience in EAI, I also deisgned and delivered a fully working Business Capability Service integration of the glossary wihtin a BI DataSet Development Framework to support Term-driven documentation at design time.\\\sep

  \multicolumn{2}{r}{\footnotesize\itshape (cont. on the next page)}\\\sep
  \newpage

  \joblog{Data Vault Architect}{Aerospace Manufacturing Company}{Brussels}{2018-03-01}{2019-01-31}
    &\sotagtech{Data Vault} \sotagtech{Star Schema} \sotagtech{SQL Server} \sotagtech{Docker} \sotagtech{Datavault Builder} \sotagtech{QlikSense} \sotagtech{PowerShell}\\&\\
    &Designed, engineered and delivered a full working agile data management platform in 4 months.\\&\\
    &The Solution server 3 use cases such as on-time delivery, supply chain issue tracking and task pointing performance.\\\sep
  
  \joblog{Data Vault Architect}{Insurance Company}{Brussels}{2014-11-01}{2018-03-31}
    &\sotagtech{Data Vault} \sotagtech{Star Schema} \sotagtech{SQL Server} \sotagtech{Qlik View} \sotagtech{SSIS} \sotagtech{C\# .NET} \sotagtech{SQL Agent}\\&\\
    &...\\\sep

  \joblog{Solution Architect \& Engineer}{Non-profit organisation}{Brussels}{2013-07-01}{2013-08-01}
    &\sotagtech{MS Access} \sotagtech{Qlik View} \sotagtech{Business Objects} \sotagtech{VBScript}\\&\\
    &...\\\sep

  \joblog{Data Vault Architect}{International Bank}{Brussels}{2012-10-01}{2013-07-01}
    &\sotagtech{SQL Server} \sotagtech{SSIS} \sotagtech{Business Objects} \sotagtech{VBA} \sotagtech{C\#} \sotagtech{VBScript} \sotagtech{Control-M}\\&\\
    &...\\\sep
  
  \joblog{Intern, Developer}{Edesign}{Ophain}{2005-02-01}{2005-05-01}
    &\sotag{java} \sotag{PHP} \sotag{SOA}\\&\\
    &Study, design and implementation of a "PermissionService" in the core company's product.\\\sep
  
\end{longtable}

\section{Data Proficiency}
\begin{tabular}{rl}
  \textsc{Data Modeling:}& Conceptual | Entity-Relationship (3NF) | Data Vault | \\
  &Dimensional (Star \& Snowflake) | Canonical \\
  \textsc{RDBMS:}& SQL Server | Oracle | MariaDB/MySQL | PostgreSQL \\
  \textsc{Documents:}& JSON | XML | XSD | CSV | DTD \\
  \textsc{DWH Automation:}& Data Vault Builder, WhereScape RED \\
  \textsc{ETL:}& SSIS | Talend\\
  \textsc{Data Visulisation:}& QlikSense | QlikView | Business Objects \\
  \textsc{API:}& REST | RAML | SOAP | WSDL | Swagger | Postman\\
  \textsc{IDE:}& Visual Studio | VS Code | Eclipse\\
  \textsc{Tools:}& Jira | Confluence | GitLab\\
  \textsc{Version Control:}& Git \\
  \textsc{O/S / Virtualization:}& Windows | Linux | vSphere | Docker\\
  \textsc{Scripting}:&Bash | PowerShell | VBScript \\
  \textsc{Programming}:&C\# | Java | Python\\
\end{tabular}

\section{Skills and Acomplishments}
\begin{tabular}{rl}
  \textsc{Languages:}& French, English, Dutch\\
\end{tabular}

\section{Specializations}
\begin{tabular}{rl}
  \textsc{Courses:}
  &Collibra Expert I | Online (Credential ID: CELIC-20190131-1072)\\
  &Certified Data Vault 2 Practioner (CDVP2) | Dan Lindstedt (\href{http://keyldv.com/}{certificate})\\
  &Business Information using RED on SQL Server | WhereScape
\end{tabular}

\end{document}
